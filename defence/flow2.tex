\begin{tikzpicture}[
  scale=0.5,
  line width=0.20mm,
  baseline=-0.1cm
  >=stealth,
  scale=1.00,
  thick,
  ]

  \tikzmath{
    \radius    = 0.72; % radius of right circle
    \dotradius = 0.07; % radius of the circles that make the dots
  }

  % define the distances between the Feynman diagrams
  \coordinate   (dist6) at ( 0.0,  0.0);

  %--------------
  %  DIAGRAM 6  |
  %--------------

  % Loop composed out of two segments + annotation
  \draw[double] ($ (dist6) + (-\radius, 0) $) arc (180:   0: \radius);
  \draw[double] ($ (dist6) + (-\radius, 0) $) arc (180: 360: \radius);

  % External leg incoming from the left
  \draw[TT]    ($ (dist6) + (-1.0*\radius, 0) $) -- ($ (dist6) + (-2.5*\radius, 0) $)
    node[above, midway, inner sep=2mm] {$\hTT$};

  % Two external legs incoming from the right: above, below
  \draw        ($ (dist6) + ( 1.0*\radius, 0) $) -- ($ (dist6) + (1.953*\radius, 0.550*\radius) $)
    node[above, midway, inner sep=2mm] {$\varphi$};
  \draw        ($ (dist6) + ( 1.0*\radius, 0) $) -- ($ (dist6) + (1.953*\radius,-0.550*\radius) $)
    node[below, midway, inner sep=2mm] {$\varphi$};

  % Two dots for vertices
  \draw[fill]  ($ (dist6) + (-\radius, 0) $) circle [radius=\dotradius];
  \draw[fill]  ($ (dist6) + ( \radius, 0) $) circle [radius=\dotradius];
\end{tikzpicture}
