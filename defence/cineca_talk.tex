% TODO Adjust the title to match the one we specified in the application.

\documentclass{beamer}

\usepackage{amsmath,amssymb,amsfonts}
\usepackage{array}
\usepackage[british]{babel}
\usepackage{bbm}
\usepackage{bbold}
\usepackage{beamerthemesplit}
\usepackage[bf]{caption}
\usepackage{cite}
\usepackage{color}
\usepackage{enumerate}
\usepackage{eurosym}
\usepackage{graphicx}
\usepackage{rotating}
\usepackage{slashed}
\usepackage[utf8]{luainputenc}
\usepackage{xcolor}
\usepackage{array}
\usepackage{listings}

\usepackage{pgfplots}
\pgfplotsset{compat=1.10}

% My fonts:
% ---------
\usepackage{libertine}
\usefonttheme[onlymath]{serif}

\usepackage[activate]{microtype}

%----------------------------------------------------------------------------------------
% C++ STYLES
%----------------------------------------------------------------------------------------

\definecolor{codegreen}{rgb}{0,0.6,0}
\definecolor{codegray}{rgb}{0.5,0.5,0.5}
\definecolor{codepurple}{rgb}{0.58,0,0.82}
\definecolor{backcolour}{rgb}{0.95,0.95,0.92}
\lstdefinestyle{mystyle}{
	language=C++,
  % backgroundcolor=\color{backcolour},
  commentstyle=\color{codegreen},
  keywordstyle=\color{magenta},
  numberstyle=\tiny\color{codegray},
  stringstyle=\color{codepurple},
  basicstyle=\tiny\ttfamily,
  breakatwhitespace=false,
  breaklines=true,
  captionpos=b,
  keepspaces=true,
  % numbers=left,
  numbersep=5pt,
  showspaces=false,
  showstringspaces=false,
  showtabs=false,
  tabsize=2,
        keywords={var, func, extends},
}
\lstset{style=mystyle}

%----------------------------------------------------------------------------------------
% OTHER STUFF
%----------------------------------------------------------------------------------------

\DeclareMathOperator{\tr}{tr}
\makeatletter
\def\amsbb{\use@mathgroup \M@U \symAMSb}
\makeatother

\usetheme{Boadilla}

\setbeamertemplate{section in toc}[sections numbered]
\setbeamertemplate{blocks}[rounded][shadow=false]
\setbeamertemplate{itemize items}[default]
\beamertemplatenavigationsymbolsempty
\setbeamercolor{footline}{bg=black}
\setbeamertemplate{footline}{%
  \raisebox{8pt}{\makebox[\paperwidth]{\hfill\makebox[15pt]{{\tiny\insertframenumber}}\qquad}}
}

\definecolor{myred}{RGB}{176,50,50}
\definecolor{myblue}{RGB}{50,50,176}
\definecolor{mygreen}{RGB}{60,166,60}

\newcommand\blfootnote[1]{%
  \begingroup
    \renewcommand\thefootnote{}\footnote{#1}%
    \addtocounter{footnote}{-1}%
    \endgroup
}

%\usepackage{pgfpages}
%\setbeameroption{show notes on second screen}


\begin{document}

\allowdisplaybreaks[1]

\title{Lattice QCD Stencils on Intel Xeon Phis}
\author{Peter Labus \and Martin Ueding \\[2ex] University of Bonn, Germany}
  \date{Cineca \\ 03/23/2017}

  %%%%%%%%%%%%%%%%%%%%%%%%%%%%%%%%%%%%%%%%%%
  %%%   TITLE SLIDE
  %%%%%%%%%%%%%%%%%%%%%%%%%%%%%%%%%%%%%%%%%%

  \begin{frame}
    \titlepage
  \end{frame}

  %%%%%%%%%%%%%%%%%%%%%%%%%%%%%%%%%%%%%%%%%%
  %%%   SLIDE 1
  %%%%%%%%%%%%%%%%%%%%%%%%%%%%%%%%%%%%%%%%%%

  \begin{frame}
    \frametitle{LQCD \& the r\^ole of the \textit{dslash} stencil}

    \begin{itemize}
      \item  Calculate integrals of the form
        \begin{align*}
          \int \mathcal D \Phi \; f[\Phi] \, \mathrm e^{-S_\text{QCD}[\Phi]}
        \end{align*}
        after \textit{discretising} space-time through a lattice
        \vfill

      \item Algorithmic layers of LQCD:
        \begin{enumerate}
          \item Hybrid Monte Carlo
          \item Efficient Krylov Solvers for $Mx=b$ in Molecular Dynamics
          \item BLAS linear algebra
        \end{enumerate}
        \vfill

      \item Most \textit{expensive} part:\\[1mm]
        \hspace{2mm} Matrix times Vector (very large \& sparse),\\
        \hspace{2mm} described by \textbf{nearest-neighbour stencil}
        \vfill
    \end{itemize}

  \end{frame}

  %%%%%%%%%%%%%%%%%%%%%%%%%%%%%%%%%%%%%%%%%%
  %%%   SLIDE 2
  %%%%%%%%%%%%%%%%%%%%%%%%%%%%%%%%%%%%%%%%%%

  \begin{frame}
    \frametitle{The QPhiX library (mainly: Bálint Joó, Jefferson Lab)}

    \begin{itemize}
      \item Aim: provide stencil operations for \textbf{general vector machines}
        \vfill
      \item Parallelised via QMP + OpenMP + SIMD (vector intrinsics)
        \vfill
      \item C++11 template library with external C++ \textbf{code generator}\\
        ($\sim25k$ lines of code + $10k$ testing \& timing)
        \vfill
      \item Implements
        \begin{enumerate}
          \item Wilson / Wilson-Clover \textit{dslash} stencils
          \item BLAS linear algebra
          \item Krylov Solvers (CG, BiCGStab, Mixed Precision, MultiCG)
        \end{enumerate}
        \vfill
      \item Common template parameters:
        \begin{enumerate}
          \item \texttt{typename FT}
          \item \texttt{int VECLEN}
          \item \texttt{int SOALEN}
          \item \texttt{bool COMPRESS12}
        \end{enumerate}
        \vfill
    \end{itemize}

  \end{frame}

  %%%%%%%%%%%%%%%%%%%%%%%%%%%%%%%%%%%%%%%%%%
  %%%   SLIDE 3
  %%%%%%%%%%%%%%%%%%%%%%%%%%%%%%%%%%%%%%%%%%

  \begin{frame}
    \frametitle{The \textit{dslash} stencil in action}

    \begin{align*}
      \nonumber
      \chi^a_\alpha(x)
      % &= \sum_{y \in \Lambda} \sum_{b = 0}^2 \sum_{\beta = 0}^3 \slashed D^{ab}_{\alpha \beta}(x,y) \psi^b_\beta(y) \\
      &= \sum_{b = 1}^3 \sum_{\beta = 1}^4 \sum_{\mu=1}^4
      \bigg[
        U^{ab}(x,x+\hat\mu) \; P^{-\, \mu}_{\alpha\beta} \; \psi^b_\beta(x+\hat\mu) + \\
        \nonumber
        &\hspace{25.0mm}
        U^{\dagger\, ab}(x,x-\hat\mu) \; P^{+\,\mu}_{\alpha\beta} \; \psi^b_\beta(x-\hat\mu)
      \bigg]
    \end{align*}

    \vfill
    For each site $x$ of the lattice:
    \vfill
    \begin{enumerate}
      \item Project on \textit{half} the spinor components via $P^{\pm\,\mu}_{\alpha\beta}$
      \item Hermitian Matrix Multiplication ($3 \times 3$ complex)
      \item Reconstruct and accumulate 8 contributions
    \end{enumerate}
    % \begin{itemize}
    %   \item $x,y$: points/sites of the lattice $\Lambda$\\
    %   \item $\psi$: input \textbf{spinor}: $3 \times 4 \times 2 = 24$ components per site\\
    %   \item $\chi$: output spinor\\
    %   \item $U$: \textbf{gauge} or link variables: $3 \times 3 \times 2 = 18$ components per site\\
    %   \item $\gamma_\mu$: 4 constant $4 \times 4$ complex matrices
    %   \item $\hat \mu$: 4 unit vectors (one for each space-time dimension)
    % \end{itemize}

  \end{frame}

  %%%%%%%%%%%%%%%%%%%%%%%%%%%%%%%%%%%%%%%%%%
  %%%   SLIDE 4
  %%%%%%%%%%%%%%%%%%%%%%%%%%%%%%%%%%%%%%%%%%

  \begin{frame}
    \frametitle{Arithmetic Intensities}
    \vspace{-5mm}
    \footnotesize

    \begin{align*}
      I_A = \textrm{FLOPs per Byte of moved data [here: single precision]}
    \end{align*}

    \begin{columns}[t]
    \begin{column}{0.49\linewidth}
      \begin{block}{Our Stencil Operations}
      \textbf{\textit{dslash} stencil:} \\ $\slashed D \,x$ \hfill 1320 FLOPs per site
      \begin{align*}
        I_A = 1.06 \; \textrm{FLOP/Byte}
      \end{align*}

      \textbf{\textit{dslash} stencil with clover term:} \\ $A^{-1}  \slashed D \, x$ \hfill 1872 FLOPs per site
      \begin{align*}
        I_A = 1.17 \; \textrm{FLOP/Byte}
      \end{align*}
      \end{block}
    \end{column}

    \hfill
    \begin{column}{0.49\linewidth}
      \begin{block}{The Hardware Peak}
      \textbf{Theoretical Peaks on KNL / KNC:}
      \begin{align*}
        I_A = 15.4 \, / \, 6.3 \; \textrm{FLOP/Byte}
      \end{align*}
      \vfill
      \end{block}
    \end{column}
    \end{columns}

    \vfill
    {
      \large
      \centering \textbf{All routines are memory bandwidth bound}\\
    }

    \vfill
    Combining these intensities with a simple hardware model and the (measured) memory bandwidth
    gives us a good \textit{rooftop model} \dots

  \end{frame}

  %%%%%%%%%%%%%%%%%%%%%%%%%%%%%%%%%%%%%%%%%%
  %%%   SLIDE 5
  %%%%%%%%%%%%%%%%%%%%%%%%%%%%%%%%%%%%%%%%%%

  \begin{frame}
    \frametitle{Intel Xeon Phi Knights Landing Hierarchies}

    \begin{columns}[T]
      \begin{column}{.5\textwidth}
        \vspace{3mm}
        {
          \center Peak Performance versus
          Memory Bandwidth:
        }
        \vspace{3mm}
        \begin{itemize}
          \item up to 72 cores @ 1.5 GHz\\ (3.5 TFLOP/s)
          \item 2D Mesh: 700 GB/s
          \item 8 GB MCDRAM: 450 GB/s
          \item 384 GB DDR-4: 90 GB/s
        \end{itemize}
      \end{column}
      \begin{column}{.5\textwidth}
        \begin{center}
          \includegraphics[width=0.9\textwidth]{knl.eps}
        \end{center}
      \end{column}
    \end{columns}

  \end{frame}

  %%%%%%%%%%%%%%%%%%%%%%%%%%%%%%%%%%%%%%%%%%
  %%%   SLIDE 6
  %%%%%%%%%%%%%%%%%%%%%%%%%%%%%%%%%%%%%%%%%%

  \begin{frame}
    \frametitle{General Programming Implications}

    Three Guidelines...
    \begin{enumerate}
      \item \textbf{Scaling}
      \item \textbf{Vectorisation}
      \item \textbf{Data Locality \& Cache Re-use}
    \end{enumerate}
    \vfill

    ... and how they are implemented in QPhiX
    \begin{enumerate}
      \item \textbf{OpenMP thread scheduling} for lattice traversal
      \item Multi-ISA \textbf{C++ code generator} using Vector Intrinsics
      \item \textbf{Cache blocking} into L2 and \textbf{Structures of Arrays} data layout (\textit{tiles})
    \end{enumerate}
    \vfill

    \textbf{Optimisations in QPhiX-codegen:}\\
    L1 \& L2 software prefetches,
    dimensions of tiles (\texttt{SOALEN}),\\
    streaming stores, gather/scatter instructions, \dots

  \end{frame}

  %%%%%%%%%%%%%%%%%%%%%%%%%%%%%%%%%%%%%%%%%%
  %%%   SLIDE 7
  %%%%%%%%%%%%%%%%%%%%%%%%%%%%%%%%%%%%%%%%%%

  \begin{frame}
    \frametitle{Data layout \& Cache blocking}

    \begin{itemize}
      \item Elementary vector elements are \textit{Structures of Arrays} (SoA):\\
        \texttt{typedef FT FourSpinorBlock[3][4][2][SOALEN];}
        \vfill

      \item \texttt{SOALEN} may be a factor of $\texttt{VECLEN} = \texttt{ngy} * \texttt{SOALEN}$:\\
        form \textit{tiles} in X-Y-plane (Xeon Phi SP: $16\times1$, $8\times2$, $4\times4$)
    \end{itemize}

    \begin{columns}

      \begin{column}{.4\textwidth}
        \begin{center}
          \begin{itemize}
            \item Then divide lattice in blocks:
              \begin{description}
                \item[X]    \texttt{SOALEN}\\
                \item[Y, Z] \texttt{By}, \texttt{Bz}\\[2mm]
              \end{description}
              such that 3 T-Slices fit into L2
              \vfill
          \end{itemize}
        \end{center}
      \end{column}

      \begin{column}{.6\textwidth}
        \begin{center}
\begin{tikzpicture}[scale=0.3]

    \fill[fill=black!20] (0, 0) rectangle +(8, 1);
    \fill[fill=black!10] (0, 1) rectangle +(8, 1);

    \draw[dotted] (0, 0) grid (16, 8);

    \draw[black!30, xstep=4, ystep=2] (0, 0) grid (16, 8);

    \draw[thick, xstep=8, ystep=4] (0, 0) grid (16, 8);
    \draw[thick] (0, 0) rectangle (16, 8);

    \draw[|-|] (0, -0.5) -- (8, -0.5) node[below, midway] {\texttt{SOALEN}};
    \draw[|-|] (-0.5, 0) -- (-0.5, 2) node[left, midway] {\texttt{ngy}};
    \draw[|-|] (-0.5, 4) -- (-0.5, 8) node[left, midway] {\texttt{By}};
    \draw[|-|] (16.5, 0) -- (16.5, 8) node[right, midway] {\texttt{Ly}};
    \draw[|-|] (0, 8.5) -- (16, 8.5) node[above, midway] {\texttt{Lx}};

          \end{tikzpicture}
        \end{center}
      \end{column}

    \end{columns}

  \end{frame}

  %%%%%%%%%%%%%%%%%%%%%%%%%%%%%%%%%%%%%%%%%%
  %%%   SLIDE 8
  %%%%%%%%%%%%%%%%%%%%%%%%%%%%%%%%%%%%%%%%%%

  \begin{frame}
    \frametitle{Heuristic Load Balancing}


    \begin{columns}
      \begin{column}{.45\linewidth}
        \includegraphics[height=0.85\textheight]{phases-crop.png}
      \end{column}
      \begin{column}{.50\linewidth}
        3.5 Dimensional Blocking:

        \begin{itemize}
          \item Each core processes one block per phase
          \item If fewer blocks than core, split in T-direction
          \item Process remainder after maximum number of splits
        \end{itemize}
      \end{column}
    \end{columns}

  \end{frame}

  %%%%%%%%%%%%%%%%%%%%%%%%%%%%%%%%%%%%%%%%%%
  %%%   SLIDE 9
  %%%%%%%%%%%%%%%%%%%%%%%%%%%%%%%%%%%%%%%%%%

  \begin{frame}
    \frametitle{QPhiX-codegen}

    \begin{itemize}

      \item Three abstract classes:
        \begin{enumerate}
          \item \texttt{Instruction}
          \item \texttt{Address}
          \item \texttt{FVec}
        \end{enumerate}
        \vfill

      \item First two classes: \;\; \texttt{std::string serialize(void)}
        \vfill

      \item Concrete implementations of \texttt{serialize} determine the ISA\\

    \lstinputlisting[language=C++, basicstyle=\footnotesize\ttfamily]{fma-avx512.cpp}

        \vfill

      \item Save Instructions into \texttt{InstVector}'s and dump to file

    \end{itemize}

  \end{frame}

  %%%%%%%%%%%%%%%%%%%%%%%%%%%%%%%%%%%%%%%%%%
  %%%   SLIDE 10
  %%%%%%%%%%%%%%%%%%%%%%%%%%%%%%%%%%%%%%%%%%

  \begin{frame}[fragile]
    \frametitle{Code Example: Hermitian Matrix Multiplication}
    \small

    \lstinputlisting[language=C++]{clover.cc}
\end{frame}

  %%%%%%%%%%%%%%%%%%%%%%%%%%%%%%%%%%%%%%%%%%
  %%%   SLIDE 13
  %%%%%%%%%%%%%%%%%%%%%%%%%%%%%%%%%%%%%%%%%%

  \begin{frame}
    \frametitle{Optimisation Features}


    \begin{columns}
      \begin{column}{0.4\linewidth}
        \begin{block}{QPhiX Code Generator}
          \begin{itemize}
            \item L1 Prefetches
            \item L2 Prefetches
            \item Streaming Stores
            \item Gauge Packing
            \item Barriers
          \end{itemize}
        \end{block}

        \begin{block}{QPhiX User Library}
          \begin{itemize}
            \item Floating point type
            \item SoA length
            \item Gauge Compression
          \end{itemize}
        \end{block}
      \end{column}
      \begin{column}{0.6\linewidth}
        \begin{tikzpicture}[scale=0.3]
          \fill[fill=black!20] (0, 0) rectangle +(8, 1);
          \fill[fill=black!10] (0, 1) rectangle +(8, 1);

          \draw[dotted] (0, 0) grid (16, 8);

          \draw[black!30, xstep=4, ystep=2] (0, 0) grid (16, 8);

          \draw[thick, xstep=8, ystep=4] (0, 0) grid (16, 8);
          \draw[thick] (0, 0) rectangle (16, 8);

          \draw[|-|] (0, -0.5) -- (8, -0.5) node[below, midway] {\texttt{SOALEN}};
          \draw[|-|] (-0.5, 0) -- (-0.5, 2) node[left, midway] {\texttt{ngy}};
          \draw[|-|] (-0.5, 4) -- (-0.5, 8) node[left, midway] {\texttt{By}};
          \draw[|-|] (16.5, 0) -- (16.5, 8) node[right, midway] {\texttt{Ly}};
          \draw[|-|] (0, 8.5) -- (16, 8.5) node[above, midway] {\texttt{Lx}};
        \end{tikzpicture}
      \end{column}
    \end{columns}


  \end{frame}

  \begin{frame}
    \frametitle{Optimisation Features on KNC vs. KNL}
    \framesubtitle{ Single precision, $\texttt{SOALEN}=16$}

    \begin{columns}[T]
      \begin{column}{.5\textwidth}
        \begin{center}
          KNC:\\
      \begin{tikzpicture}[scale=0.6]
        \begin{axis}[
            xtick=data,
            xticklabels={
              {\parbox{7em}{vector + packed \\ + streaming}},
           + L1,
           + L2,
           + L1 and L2,
+ Barrier,
+ SOA=8,
+ compression
            },
            x tick label style={rotate=-35,anchor=north west},
                  ylabel={GFlop/s},
                  ymin=0,
                  ymax=470,
                  ybar,
                  legend pos=outer north east,
                  ymajorgrids=true,
                ]

              \addplot coordinates
              {
                (1, 172)
                (2, 144)
                (3, 176)
                (4, 191)
                (5, 199)
                (6, 234)
                (7, 286)
              };

        \end{axis}
      \end{tikzpicture}

          \vfill
        \end{center}
      \end{column}
      \begin{column}{.5\textwidth}
        \begin{center}
          KNL (DEEP):\\
      \begin{tikzpicture}[scale=0.6]
        \begin{axis}[
            xtick=data,
            xticklabels={
vector,
+ packed gauges,
+ streaming stores,
(3) + L1,
(3) + L2,
(3) + L1 and L2,
(3) + Barrier,
+ SOA=8,
+ compression,
            },
            x tick label style={rotate=-35,anchor=north west},
                  ylabel={GFlop/s},
                  ymin=0,
                  ymax=470,
                  ybar,
                  legend pos=outer north east,
                  ymajorgrids=true,
                ]

              \addplot+[error bars/.cd, y dir=both, y explicit] coordinates
              {
                (1, 362.5) +- (0, 18.4)
                (2, 384.3) +- (0, 19.3)
                (3, 380.1) +- (0, 18.6)
                (4, 377.5) +- (0, 18.6)
                (5, 344.9) +- (0, 14.3)
                (6, 359.5) +- (0, 13.5)
                (7, 381.2) +- (0, 26.9)
                (8, 386.9) +- (0, 21.0)
                (9, 434.2) +- (0, 14.5)
              };

        \end{axis}
      \end{tikzpicture}
        \end{center}
      \end{column}
    \end{columns}

    \begin{small}
    \begin{itemize}
      \item KNL natively supports prefetches \& streaming stores much better than KNC
      \item Visible benefits from data layout tuning (packing \& SoA length)
      \item Greatest benefits from algorithmic improvements, though
    \end{itemize}
    \end{small}

  \end{frame}

  \begin{frame}
    \frametitle{Single Node Performance}

      \begin{columns}[t]
        \begin{column}{0.4\linewidth}
          \begin{tikzpicture}
            \begin{axis}[
                title=DEEP,
                width=\linewidth,
                height=0.5\textheight,
                xmin=1.5,
                xmax=3.5,
                ymin=0,
                ymax=470,
                xtick={2, 3},
                xticklabels={4, 8},
                xlabel={SoA length},
                ylabel={GFlop/s},
                ybar,
                ymajorgrids=true,
              ]

              \addplot+[error bars/.cd, y dir=both, y explicit] coordinates {
                ( 2.0, 299.804833333) +- (0, 10.4998327882)
                ( 3.0, 437.045666667 ) +- (0, 9.28507893846) };
              \addplot+[error bars/.cd, y dir=both, y explicit] coordinates {
                ( 2.0, 371.738) +- (0, 6.58345238394)
                ( 3.0, 421.222) +- (0, 5.61352336748) };

            \end{axis}
          \end{tikzpicture}
        \end{column}

        \begin{column}{0.6\linewidth}
          \begin{tikzpicture}
            \begin{axis}[
                title=Marconi A2,
                width=0.7\linewidth,
                height=0.5\textheight,
                xmin=1.5,
                xmax=3.5,
                ymin=0,
                ymax=470,
                xtick={2, 3},
                xticklabels={4, 8},
                xlabel={SoA length},
                ybar,
                legend pos=outer north east,
                ymajorgrids=true,
              ]

              \addplot+[error bars/.cd, y dir=both, y explicit] coordinates {
                ( 2.0 , 217.841833333) +- (0, 0.993589751813)
                ( 3.0 , 363.165083333) +- (0, 2.7395950569) };
              \addlegendentry{ Half }
              \addplot+[error bars/.cd, y dir=both, y explicit] coordinates {
                ( 2.0 , 306.195333333) +- (0, 1.95385987405)
              ( 3.0 , 315.29625) +- (0, 1.86925662633) };
              \addlegendentry{ Single }
              \addplot+[error bars/.cd, y dir=both, y explicit] coordinates {
                ( 2.0 , 122.167083333) +- (0, 0.678511273327)
              ( 3.0 , 114.3745) +- (0, 0.753930501697) };
              \addlegendentry{ Double }

            \end{axis}
          \end{tikzpicture}
        \end{column}
      \end{columns}

          One KNL, one MPI rank, $64 \times 4$ threads, $L = 48$, $T = 96$

      %Half prec is not faster in hardware, is good for bandwidth limited programs (like ours)
    \end{frame}

  \begin{frame}
    \frametitle{Multi Node Performance}
      \framesubtitle{Best Known KNL Parameters}

      \begin{columns}[t]
        \begin{column}{0.45\linewidth}
          \begin{block}{QPhiX Configuration}
            \begin{itemize}
              \item SOALEN = 8
              \item Block size = 4
              \item Single precision
              \item 12-Parameter gauge compression
              \item Lattice sizes $32^3 \times 64$, $48^3 \times 96$, and $64^3 \times 128$
              \item Error bar denotes standard error with 12 repetitions
            \end{itemize}
          \end{block}

          \hfill

        \end{column}
        \begin{column}{0.45\linewidth}
          \begin{block}{Hardware Configuration}
            \begin{itemize}
              \item Memory mode: Cache
              \item Cluster mode: Quadrant
              \item One MPI rank per KNL, using 68 threads
              \item 8, 16, 32, and 64 KNL
              \item Different rank geometries ($g_1 g_2 g_3 g_4 = \text{\# ranks}$)
              \item Runs performed 2017-03-14 (last Tuesday)
            \end{itemize}
          \end{block}

        \end{column}
      \end{columns}

  \end{frame}

  \begin{frame}
      \frametitle{Multi Node Performance}
      \framesubtitle{Strong Scaling}

      \begin{columns}
        \begin{column}{0.5\linewidth}
             
      \begin{tikzpicture}
          \begin{axis}[
              title=JURECA,
              width=\linewidth,
              height=0.6\textheight,
                  xmin=0,
                  ymin=0,
                  xlabel={Number of Nodes},
                  ylabel={TFlop/s Total},
                  %legend pos=outer north east,
                  grid=major,
              ]
              \addplot+[only marks, mark=*, error bars/.cd, y dir=both, y explicit]
              table[y error index=2] {jureca-strong-64.tsv};
              %\addlegendentry{$L = 64$}
              \addplot+[only marks, mark=*, error bars/.cd, y dir=both, y explicit]
              table[y error index=2] {jureca-strong-48.tsv};
              %\addlegendentry{$L = 48$}
              \addplot+[only marks, mark=*, error bars/.cd, y dir=both, y explicit]
              table[y error index=2] {jureca-strong-32.tsv};
              %\addlegendentry{$L = 32$}
          \end{axis}
      \end{tikzpicture}

         \end{column}
        \begin{column}{0.5\linewidth}
             
      \begin{tikzpicture}
          \begin{axis}[
              title=Marconi A2,
              width=\linewidth,
              height=0.6\textheight,
                  xmin=0,
                  ymin=0,
                  xlabel={Number of Nodes},
                  %ylabel={GFlop/s Total},
                  %legend pos=outer north east,
                  grid=major,
                  xtick=data,
              ]
              \addplot+[only marks, mark=*, error bars/.cd, y dir=both, y explicit]
              table[y error index=2] {strong-64.tsv};
              %\addlegendentry{$L = 64$}
              \addplot+[only marks, mark=*, error bars/.cd, y dir=both, y explicit]
              table[y error index=2] {strong-48.tsv};
              %\addlegendentry{$L = 48$}
              \addplot+[only marks, mark=*, error bars/.cd, y dir=both, y explicit]
              table[y error index=2] {strong-32.tsv};
              %\addlegendentry{$L = 32$}
          \end{axis}
      \end{tikzpicture}

         \end{column}
      \end{columns}
      
        \vfill

      Lattice sizes: Blue 64, Red 48, Gold 32.

  \end{frame}

  \begin{frame}
      \frametitle{Multi Node Performance}
      \framesubtitle{Weak Scaling}

      \begin{columns}
        \begin{column}{0.5\linewidth}
            
      \begin{tikzpicture}
          \begin{axis}[
              title=JURECA,
              width=\linewidth,
              height=0.6\textheight,
                  xmin=0,
                  ymin=0,
                  xlabel={Number of Nodes},
                  ylabel={GFlop/s per Node},
                  %legend pos=outer north east,
                  grid=major,
              ]
              \addplot+[only marks, mark=*, error bars/.cd, y dir=both, y explicit]
              table[y error index=2] {jureca-weak-64.tsv};
              %\addlegendentry{$L = 64$}
              \addplot+[only marks, mark=*, error bars/.cd, y dir=both, y explicit]
              table[y error index=2] {jureca-weak-48.tsv};
              %\addlegendentry{$L = 48$}
              \addplot+[only marks, mark=*, error bars/.cd, y dir=both, y explicit]
              table[y error index=2] {jureca-weak-32.tsv};
              %\addlegendentry{$L = 32$}
          \end{axis}
      \end{tikzpicture}
        \end{column}

        \begin{column}{0.5\linewidth}
            
      \begin{tikzpicture}
          \begin{axis}[
              title=Marconi A2,
              width=\linewidth,
              height=0.6\textheight,
                  xmin=0,
                  ymin=0,
                  xlabel={Number of Nodes},
                  %ylabel={GFlop/s per Node},
                  %legend pos=outer north east,
                  grid=major,
                  xtick=data,
              ]
              \addplot+[only marks, mark=*, error bars/.cd, y dir=both, y explicit]
              table[y error index=2] {weak-64.tsv};
              %\addlegendentry{$L = 64$}
              \addplot+[only marks, mark=*, error bars/.cd, y dir=both, y explicit]
              table[y error index=2] {weak-48.tsv};
              %\addlegendentry{$L = 48$}
              \addplot+[only marks, mark=*, error bars/.cd, y dir=both, y explicit]
              table[y error index=2] {weak-32.tsv};
              %\addlegendentry{$L = 32$}
          \end{axis}
      \end{tikzpicture}
        \end{column}
          
      \end{columns}

      
        \vfill

      Lattice sizes: Blue 64, Red 48, Gold 32.
  \end{frame}

  \begin{frame}
    \frametitle{Benefits of backporting to Haswell}
    \framesubtitle{Solving a linear system of equations}

    \begin{columns}
      \begin{column}{0.5\linewidth}
          \begin{tikzpicture}
            \begin{axis}[
                width=\linewidth,
                %height=0.3\textheight,
                ymin=0,
                xmin=0.5,
                xmax=2.5,
                xtick={1, 2},
                xticklabels={Chroma, QPhiX},
                xlabel={Implementation},
                ylabel={Time to Solution (s)},
                ybar,
                ymajorgrids=true,
                ytick=data,
              ]

              \addplot coordinates {
                ( 1.0 , 6.073364 )
                ( 2.0 , 0.590205 )
              };


            \end{axis}
          \end{tikzpicture}

%          \begin{tikzpicture}
%            \begin{axis}[
%                title={Solve},
%                width=\linewidth,
%                height=0.3\textheight,
%                ymin=0,
%                xmin=0.5,
%                xmax=2.5,
%                xtick={1, 2},
%                xticklabels={Chroma, QPhiX},
%                xlabel={Implementation},
%                ylabel={Seconds},
%                ybar,
%                ymajorgrids=true,
%                ytick=data,
%              ]
%
%              \addplot coordinates {
%                ( 1.0 , 1079.359599 )
%                ( 2.0 , 748.634722 )
%              };
%
%            \end{axis}
%          \end{tikzpicture}
          
      \end{column}

      \begin{column}{0.5\linewidth}
        \begin{itemize}
          \item 81 Nodes: \\
          2 Xeon E5-2680 v3 @ 2.50\,GHz \\ $2×12$ cores, HyperThreading unused
          \item Compiler: Intel C++ 2017

            \vspace{1.5ex}

          \item BiCGStab Solver: \\ $265 \pm 3$ iterations
        \end{itemize}
      \end{column}
    \end{columns}


    \begin{small}
      \begin{itemize}
        \item $24^3 \times 96$ lattice (almost thermalized)
      \end{itemize}
    \end{small}


  \end{frame}

  %%%%%%%%%%%%%%%%%%%%%%%%%%%%%%%%%%%%%%%%%%
  %%%   SLIDE 16
  %%%%%%%%%%%%%%%%%%%%%%%%%%%%%%%%%%%%%%%%%%

  \begin{frame}
    \frametitle{Conclusions \& Outlook}

    \begin{itemize}
        \item Enforcing SIMD with code generation appears to be a must for
          performance
        \item Optimization efforts for KNL also pay off very well for Haswell
    \end{itemize}

    \vfill

    \begin{itemize}
      \item Single node performance on Marconi A2 is 25\,\% lower than on DEEP
      \item Strong scaling on KNL inferior to Haswell
    \end{itemize}

    \vfill

    \begin{itemize}
      \item Aim: Scaling beyond 128 KNL
        \begin{itemize}
            \item Microbenchmarks
            \item Hide communication
        \end{itemize}
      \item QPhiX usable within multigrid solvers
    \end{itemize}
  \end{frame}


  \begin{frame}
    \titlepage
  \end{frame}

  \end{document}

  % vim: spell spelllang=en_gb sw=2
