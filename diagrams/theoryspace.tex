\begin{tikzpicture}[
  framed,
  inner frame sep=10pt,
  >=stealth,
  scale=1.00,
]

  \tikzmath{
    \radius    = 0.20; % radius of CFT circle
    \radiusgfp = 1.75; % radius of perturbation theory circle
    \dotradius = 0.07; % radius of FP dot
    \dotgamma  = 0.04; % radius of \Gamma dot
  }

  \coordinate (base) at ( 0.5,  0.0 );

  \coordinate (cft1)    at ( 4.0,  -0.5 );
  \coordinate (cft2)    at ( 4.5,  -6.0 );
  \coordinate (cft3)    at ( 7.0,  -5.2 );
  \coordinate (ngfp)    at ( 6.7,  -2.8 );

  \coordinate (gfp)     at ( 1.0,  -3.0 );
  \coordinate (dist1)    at ( -0.3, -0.9 );
  \coordinate (dist2)    at ( 0.35, -0.05 );
  \coordinate (dist3)    at ( 0.0, -0.35 );
  \coordinate (dist4)    at ( -0.25, -0.55 );

  \coordinate (cutoff)  at ( 6.0,  -1.5 );
  \coordinate (gammak0) at ( 4.6,  -3.5 );
  \coordinate (gamma00) at ( 1.8,  -3.7 );
  \coordinate (gamma01) at ( 3.6,  -4.7 );

  % Title
  \node[rectangle,draw] (title) at (base) {Theory Space};

  % Annotations for CFTs, GFP & NGFP
  \node[inner sep=4mm] at (cft1) [anchor=south] {CFT/NGFP};
  \node[inner sep=2mm] at (gfp) [anchor=south] {GFP};
  \node[inner sep=3mm] at (ngfp) [anchor=west]
    {$S_{\mathrm{\scriptscriptstyle{UV}}} = \Gamma_\infty$};

  % Fixed point dotes and circles
  \fill[black] (cft1) circle (\dotradius);
  \draw        (cft1) circle (\radius);
  \fill[black] (cft2) circle (\dotradius);
  \draw        (cft2) circle (\radius);
  \fill[black] (cft3) circle (\dotradius);
  \draw        (cft3) circle (\radius);
  \fill[black] (ngfp) circle (\dotradius);
  \draw        (ngfp) circle (\radius);
  \fill[black] (gfp)  circle (\dotradius);
  \draw        (gfp)  circle (\radiusgfp);

  % Additional Gamma dots
  \fill[black] (cutoff) circle (\dotgamma);
  \fill[black] (gammak0) circle (\dotgamma);
  \fill[black] (gamma00) circle (\dotgamma);
  \fill[black] (gamma01) circle (\dotgamma);

  % Additional annotations
  \node[inner sep=1mm] at (cutoff) [anchor=south west]
    {$\Gamma_{k=\Lambda_{\mathrm{\scriptscriptstyle{max}}}}$};
  \node[inner sep=0.5mm] at (gammak0) [anchor=south east]
    {$\Gamma_{k_0}$};
  \node[inner sep=1mm] at (gamma00) [anchor=north east]
    {$\Gamma$};
  \node[inner sep=1mm] at (gamma01) [anchor=north east]
    {$\Gamma$};

  % Arrow for pert theory / non-pert theory
  \node[align=center] (pert) at (0.6, -5.5) [anchor=north] {perturbation theory\\applicable};
  \draw[->] (pert) -- ($ (gfp) + (dist1)$);
  \node[align=center] (nonpert) at (7.0, -6.0) [anchor=north] {non-perturbative\\tools needed};
  \draw[->] (nonpert) -- ($ (cft2) + (dist2)$);
  \draw[->] (nonpert) -- ($ (cft3) + (dist3)$);
  \draw[->] (nonpert) -- ($ (gammak0) + (dist4)$);

  % Trajectories NGFP
  \draw[midarrow] (ngfp)    .. controls ( 6.0, -3.8 ) and ( 5.0, -2.3 ) .. (gammak0);
  \draw[midarrow] (gammak0) .. controls ( 4.5, -3.9 ) and ( 3.6, -3.9 ) .. (gamma01);
  \node[align=center] at (6.0, -3.9) {fundamental\\QFT};

  % Trajectories EFT
  \draw[midarrow] (cutoff) .. controls ( 5.0, -3.0 ) and  ( 3.0, -1.7 ) .. (gamma00)
    node[midway, above, inner sep=0.3cm] {EFT};

\end{tikzpicture}
