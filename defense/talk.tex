\documentclass{beamer}

\usepackage{amsmath,amssymb,amsfonts}
\usepackage{array}
\usepackage[british]{babel}
\usepackage{bbm}
\usepackage{bbold}
\usepackage{beamerthemesplit}
\usepackage[bf]{caption}
\usepackage{cite}
\usepackage{color}
\usepackage{enumerate}
\usepackage{eurosym}
\usepackage{graphicx}
\usepackage{rotating}
\usepackage{slashed}
\usepackage[utf8]{luainputenc}
\usepackage{xcolor}
\usepackage{array}
\usepackage{listings}
\usepackage{nicefrac} % for \nicefrac macro giving nice fractions in exponent

\usepackage{pgfplots}
\pgfplotsset{compat=1.10}

% My fonts:
% ---------
\usepackage{libertine}
\usefonttheme[onlymath]{serif}
\usepackage[activate]{microtype}

% My definitions:
% ---------------
\newcommand\GNewton{ G_{\scriptscriptstyle{\mathrm{N}}}{} }
\newcommand\SEH{ S_{\scriptscriptstyle{\mathrm{EH}}}{} }
\newcommand\SUV{ S_{\scriptscriptstyle{\mathrm{UV}}}{} }
\newcommand\Sbare{ S_{\scriptscriptstyle{\mathrm{bare}}}{} }
\newcommand\Sclass{ S_{\scriptscriptstyle{\mathrm{classical}}}{} }
\newcommand\MPl{ M_{\scriptscriptstyle{\mathrm{Pl}}}{} }
\newcommand\metric{ g_{\mu\nu} }

%----------------------------------------------------------------------------------------
% OTHER STUFF
%----------------------------------------------------------------------------------------

\makeatletter
\def\amsbb{\use@mathgroup \M@U \symAMSb}
\makeatother

\usetheme{Boadilla}

\setbeamertemplate{section in toc}[sections numbered]
\setbeamertemplate{blocks}[rounded][shadow=false]
\setbeamertemplate{itemize items}[default]
\beamertemplatenavigationsymbolsempty
\setbeamercolor{footline}{bg=black}
\setbeamertemplate{footline}{%
  \raisebox{8pt}{\makebox[\paperwidth]{\hfill\makebox[15pt]{{\tiny\insertframenumber}}\qquad}}
}

\definecolor{myred}{RGB}{176,50,50}
\definecolor{myblue}{RGB}{50,50,176}
\definecolor{mygreen}{RGB}{60,166,60}

\newcommand\blfootnote[1]{%
  \begingroup
    \renewcommand\thefootnote{}\footnote{#1}%
    \addtocounter{footnote}{-1}%
    \endgroup
}

%----------------------------------------------------------------------------------------
% DOCUMENT
%----------------------------------------------------------------------------------------

\begin{document}

%%%%%%%%%%%%%%%%%%%%%%%%%%%%%%%%%%%%%%%%%%
%%%   TITLE SLIDE
%%%%%%%%%%%%%%%%%%%%%%%%%%%%%%%%%%%%%%%%%%

\title{Matter fields in the Asymptotic Safety scenario}
\author{Peter Labus}
\institute{SISSA Trieste
  \vspace{5mm}
  \\ \vspace{8pt}PL, T. R. Morris and Z. H. Slade, Phys. Rev. D 94 (2016)
  \\ \vspace{8pt}P. Don\`a, A. Eichhorn, PL and R. Percacci, Phys. Rev. D93 (2016)
  \\ \vspace{8pt}PL, R. Percacci, G.P. Vacca, Phys. Lett.  B753 (2016)
  \\ \vspace{8pt}A. Eichhorn, PL, J. M. Pawlowski, M. Reichert, \textit{In preparation.}
  \date{14/09/2017}
}
\begin{frame}[noframenumbering,plain]
  \titlepage
\end{frame}

%----------------------------------------------------------------------------------------
%   Section 1 --- Quantum Gravity
%----------------------------------------------------------------------------------------

\section{Asymptotically safe quantum gravity}
\fontsize{10pt}{7.2}\selectfont

%%%%%%%%%%%%%%%%%%%%%%%%%%%%%%%%%%%%%%%%%%
%%%   SLIDE 1
%%%%%%%%%%%%%%%%%%%%%%%%%%%%%%%%%%%%%%%%%%

% TODO:
% consistent spacing

\begin{frame}
  \frametitle{Why quantum gravity?}
  \textbf{Two pressing issues:}
  \begin{enumerate}
    \item singularities in GR (black holes, big bang)
      \hspace{2cm}
      \raisebox{-.5\height}{\includegraphics[width=0.25\textwidth]{blackholes_singularity.jpg}}
    \item singularities in SM (stability of $U(1)$ and Higgs sector)
      % \hspace{1cm}
      \raisebox{-.5\height}{\includegraphics[width=0.25\textwidth]{running_alpha.png}}
  \end{enumerate}
  \vfill
  \pause
  \textbf{More theoretical issues:}\\[5pt]
  cosmological constant problem, hierarchy problem, origin of symmetries and field content,
  CP-violation, grand unification?, TOE?, \dots
\end{frame}

\addtocounter{framenumber}{-1}
\begin{frame}
  \frametitle{Why quantum gravity?}
  \textbf{Two pressing issues:}
  \begin{enumerate}
    \item singularities in GR (black holes, big bang)
      \hspace{2cm}
      \raisebox{-.5\height}{\includegraphics[width=0.25\textwidth]{blackholes_singularity.jpg}}
    \item singularities in SM (stability of $U(1)$ and Higgs sector)
      % \hspace{1cm}
      \raisebox{-.5\height}{\includegraphics[width=0.25\textwidth]{running_alpha.png}}
  \end{enumerate}
  \vfill
  \begin{center}
    \fontsize{12pt}{7.2}\selectfont
    \textbf{ Can one help to cure the other? }
  \end{center}
  \pause
  \begin{center}
    candidates:\\[5pt]
    string theory, loop quantum gravity, causal dynamical triangulation, Regge calculus,\\
    causal sets, spin foams, group field theory, \dots
  \end{center}
\end{frame}

\addtocounter{framenumber}{-1}
\begin{frame}
  \frametitle{Why quantum gravity?}
  \textbf{Two pressing issues:}
  \begin{enumerate}
    \item singularities in GR (black holes, big bang)
      \hspace{2cm}
      \raisebox{-.5\height}{\includegraphics[width=0.25\textwidth]{blackholes_singularity.jpg}}
    \item singularities in SM (stability of $U(1)$ and Higgs sector)
      % \hspace{1cm}
      \raisebox{-.5\height}{\includegraphics[width=0.25\textwidth]{running_alpha.png}}
  \end{enumerate}
  \vfill
  \begin{center}
    \fontsize{12pt}{7.2}\selectfont
    \textbf{ Can one help to cure the other? } \\[15pt]
    \textbf{ Have to go beyond QFTs? }
  \end{center}
\end{frame}

%%%%%%%%%%%%%%%%%%%%%%%%%%%%%%%%%%%%%%%%%%
%%%   SLIDE 2
%%%%%%%%%%%%%%%%%%%%%%%%%%%%%%%%%%%%%%%%%%

% TODO:
% references
% check formula
% Feynman diagram of graviton scattering
% picture of Einstein?

\begin{frame}
  \frametitle{But wait a minute...!}
  \textbf{We know (low-energy) quantum gravity:}\\[5pt]
  GR as an EFT $\rightarrow$ can calculate corrections to Newtonian potential [Donogue et al.]
  \begin{align*}
    \boxed{
      V(r) = -\frac{\GNewton m_1 m_2}{r}
      \bigg(
        1
        + 3 \, \frac{\GNewton (m_1 + m_2)}{r \, c^2}
        + \frac{41}{10 \pi} \, \frac{\GNewton \hbar}{r^2 \, c^3}
        + \dots
      \bigg)
    }
  \end{align*}
  \hfill Probably most accurate result in quantum gravity...!
  \pause
  \vfill
  \begin{itemize}
    \item well-defined QFT of gravity\\[5pt]
    \item predictive, but experimentally indistinguishable from GR\\[5pt]
    \item consistent with all available data
  \end{itemize}
\end{frame}

\addtocounter{framenumber}{-1}
\begin{frame}
  \frametitle{But wait a minute...!}
  \textbf{We know (low-energy) quantum gravity:}\\[5pt]
  GR as an EFT $\rightarrow$ can calculate corrections to Newtonian potential [Donogue et al.]
  \begin{align*}
    \boxed{
      V(r) = -\frac{\GNewton m_1 m_2}{r}
      \bigg(
        1
        + 3 \, \frac{\GNewton (m_1 + m_2)}{r \, c^2}
        + \frac{41}{10 \pi} \, \frac{\GNewton \hbar}{r^2 \, c^3}
        + \dots
      \bigg)
    }
  \end{align*}
  \hfill Probably most accurate result in quantum gravity...!
  \pause
  \vfill
  \begin{center}
    \fontsize{12pt}{7.2}\selectfont
    \textbf{ So, what's the catch? }
  \end{center}
  \pause
  \vspace{10pt}
  Expansion in $\nicefrac{k}{\MPl}$ needs $\infty$--terms in UV:\\[5pt]
  \begin{itemize}
    \item How to make predictions at the Planck scale?\\[5pt]
    \item Can we address either of our two issues?
  \end{itemize}
\end{frame}

%%%%%%%%%%%%%%%%%%%%%%%%%%%%%%%%%%%%%%%%%%
%%%   SLIDE 3
%%%%%%%%%%%%%%%%%%%%%%%%%%%%%%%%%%%%%%%%%%

% TODO:
% picture Feynman?

\begin{frame}
  \frametitle{The path integral approach}
  % \textbf{Have to make sense of}
  \begin{align*}
    \boxed{
    \int \mathcal D \Phi \; e^{i S[\Phi]}\,,
    \text{ let's try } \Phi = \metric \,.
    }
  \end{align*}
  \vfill
  % Have to specify:
  \begin{itemize}
    \item integration measure $\mathcal D \metric$ (integral over all space-times)
    \item action $S[\metric]$
  \end{itemize}
  \vfill
  \begin{enumerate}
    \item $S[\metric] = \SEH[\metric] = - \frac{1}{16 \pi \GNewton} \int \sqrt{g} \, R$
      \hspace{2pt}
      \textit{does not work} [t'Hooft, Veltman]\\[15pt]
    \item $S = \Sbare = \SUV = \, ???\,,$\\[5pt]
          $S \neq \Sclass$
  \end{enumerate}
\end{frame}

\addtocounter{framenumber}{-1}
\begin{frame}
  \frametitle{The path integral approach}
  % \textbf{Have to make sense of}
  \begin{align*}
    \boxed{
    \int \mathcal D \Phi \; e^{i S[\Phi]}\,,
    \text{ let's try } \Phi = \metric \,.
    }
  \end{align*}
  \vfill
  % Have to specify:
  \begin{itemize}
    \item integration measure $\mathcal D \metric$ (integral over all space-times)
    \item action $S[\metric]$
  \end{itemize}
  \vfill
  \begin{center}
    \fontsize{12pt}{7.2}\selectfont
    \textbf{ What is going wrong here? }
  \end{center}
  \begin{columns}[T]
    \begin{column}{.5\textwidth}
      \begin{center}
        couplings of the theory are \textbf{scale-dependent},\\[10pt]
        na\"ively $\GNewton \rightarrow \infty$ in the UV limit
      \end{center}
    \end{column}
    \begin{column}{.5\textwidth}
      \begin{center}
        vacuum polarisation in QED:
        \includegraphics[width=0.4\textwidth]{vacuum_polarisation.jpg}
      \end{center}
    \end{column}
  \end{columns}
\end{frame}

\section{Matter fields in asymptotic safety}
\section{Technical tools}
\section{Known issues}
\section{My research \& and its implications}

\begin{frame}
  \titlepage
\end{frame}

\end{document}
